%
% MpLtX --- a LaTeX Template for Modern Physics Lab
% Copyright (C) 2013 Modern Phys. Lab, School of Phys., Peking Univ.
%
%   MpLtX is a template for experiment report of Modern Physics Lab in
% Peking University. This template depends on the "revtex4.1" package from
% APS Journals <http://publish.aps.org/revtex/revtex-faq>
%
% To use this template, you should open the package download from APS Journals'
% website as above and follow instructions from the README file in the package.
%
% LaTeX is marvelous for math formulae composition. However, the script grammar
% is rather difficult to handle. Maybe at the beginning, it's convenient to
% generate a pretty document. The deeper you went, more weird grammar you got.
% Before you found out the whole fantesy-like world built by Knuth, Lamport and
% numerous contributors, you would get numerous strange errors unclearly
% reported by compiler.
%
% Anyway, a lot of people wish to find a general document system which is both
% easy to use and strong enough to conveniently DIY. Word is easy to use.
% However, Word can not produce perfect document in art --- the position and
% size are not well calculated. By the way, it's such a pain to do simple but
% repeating work in Word such as formating title, generate large data table
% and etc. These works can be easily done in LaTeX if you know a little about
% programming. HTML is a easy-to-use language to create static document. It is
% compatible on all the machines currently because all you need is a simple
% browser (Firefox, Chrome or IE). In HTML5, the latest version of HTML, you
% can do colorful presentation about the report. You can present dynamic
% figures to present your idea clearly. However, the biggest problem for HTML
% is that this never renders a beautiful math formula in a simple way. HTML
% indeed has a math engine named as MathML. But this guy is notorious for its
% unreadable script grammar. So HTML+TeX --- the project MathJax, becomes a
% candidate of our dream communication media or e-document form. However, it is
% still under development. If you are interested in Java, JLaTeXMath package may
% be also a proper one since it provides a LaTeX renderer in Java.
%
% This template is modified by students in Peking University.
%   I am Sun Sibai. Cao Chuanwu shared the draft on RenRen Network. However,
% the draft did not match the requirement at all. It seems that Cao Chuanwu
% did not modified the style from package. He just put the origin content
% into LaTeX format.
%   I changed the style to satisfy the format requirement and fixed some problem
% about the incompatibility within the packages.
%
% So, if you have suggestions, please improve this template with your power. We
% will be always glad to see our work useful, popular and wonderful!
%
% This template has been tested in TeXLive 2012 with the command:
% $ xelatex mpltx.tex
% compile twice.
%
% Anyone can modify this template, but don't forget to list the previous
% developers and add yourself in.
%
% Sun Sibai <niasw@pku.edu.cn>
% Cao Chuanwu <>
%
\documentclass[12pt, letterpaper]{article}
\usepackage[utf8]{inputenc}
\usepackage{amsmath}

\title{Title}
\author{Yuchuang \thanks{funded by ...}}
\date{December 2021}

\begin{document}
	
\begin{titlepage}
	\maketitle
\end{titlepage}


\section{Introduction}
Here is an introduction.

\section{Formulation}



\newpage
\section{Results}



\section{Conclusions}

\subsection*{Aknowledgements}


%\bibliography{apssamp}
\begin{thebibliography}{}
\bibitem{1905} (e.g.) Einstein, Albert 1905 {\it Annalen der Physik} {\bf 322}(10) 891-921.
%
\end{thebibliography}

\clearpage
\appendix
\section{Priors and Posteriors}
These part show prior and conditional posterior $X|\cdot$ for all variables $X$

\subsection{$T_0$}
Prior is multivariate normal
\begin{equation}
	\sim N\left(\mu_{0}, \mathbf{\Sigma_0}\right)
\end{equation}
Posterior is multivariate normal
\begin{equation}
	\sim N(\Psi_0V_0,\Psi_0)
\end{equation} 
where
\begin{align}
\mathbf{V}_{0}&=\mathbf{\Sigma}^{-1}\left[\alpha \mathbf{T}_{1}-\alpha(1-\alpha) \mu \mathbf{1}\right]+\mathbf{\Sigma}_{0}^{-1} \mu_{\mathbf{0}}\\
\boldsymbol{\Psi}_{0}&=\left(\alpha^{2} \boldsymbol{\Sigma}^{-1}+\boldsymbol{\Sigma}_{\mathbf{0}}^{-1}\right)^{-1}
\end{align}


\subsection{$T_k$}
Posterior is multivariate normal (here $0<k<\kappa$)
\begin{equation}
	\sim N(\Psi_kV_k,\Psi_k)
\end{equation} 
where
\begin{align}
\mathbf{V}_{k}&= \mathbf{H}_{k}^{\mathrm{T}} \mathbf{\Xi}_{k}^{-1}\left(\mathbf{W}_{k}-\mathbf{B}_{k}\right)+\mathbf{\Sigma}^{-1}\left[\alpha\left(\mathbf{T}_{k+1}+\mathbf{T}_{k-1}\right)\right.\left.+(1-\alpha)^{2} \mu \mathbf{1}\right]\\
\boldsymbol{\Psi}_{k}&=\left[\mathbf{H}_{k}^{\mathrm{T}} \boldsymbol{\Xi}_{k}^{-1} \mathbf{H}_{k}+\left(1+\alpha^{2}\right) \mathbf{\Sigma}^{-1}\right]^{-1}
\end{align}

\subsection{$T_\kappa$}

Posterior is multivariate normal
\begin{equation}
	\sim N(\Psi_kV_\kappa,\Psi_\kappa)
\end{equation} 
where
\begin{align}
\mathbf{V}_{\kappa}&=\mathbf{H}_{\kappa}^{\mathrm{T}} \Xi_{\kappa}^{-1}\left(\mathbf{W}_{\kappa}-\mathbf{B}_{\kappa}\right)+\mathbf{\Sigma}^{-1}\left[\alpha \mathbf{T}_{\kappa-1}+(1-\alpha) \mu \mathbf{1}\right]\\
\boldsymbol{\Psi}_{\kappa}&=\left(\mathbf{H}_{\kappa}^{\mathrm{T}} \boldsymbol{\Xi}_{\kappa}^{-1} \mathbf{H}_{\kappa}+\boldsymbol{\Sigma}^{-1}\right)^{-1}
\end{align}

\subsection{$\alpha$}
Prior is uniform
\begin{equation}
	\sim U(a_0,a_1)
\end{equation} 
Posterior is truncated normal
\begin{equation}
\sim N_{\left[a_{0}, a_{1}\right]}\left(\Psi_{\alpha} V_{\alpha}, \Psi_{\alpha}\right)
\end{equation}
where
\begin{align}
V_{\alpha}&=\sum_{k=1}^{\kappa}\left(\mathbf{T}_{k-1}-\mu \mathbf{1}\right)^{\mathrm{T}} \mathbf{\Sigma}^{-1}\left(\mathbf{T}_{k}-\mu \mathbf{1}\right)\\
\Psi_{\alpha}&=\left[\sum_{k=1}^{\kappa}\left(\mathbf{T}_{k-1}-\mu \mathbf{1}\right)^{\mathrm{T}} \mathbf{\Sigma}^{-1}\left(\mathbf{T}_{k-1}-\mu \mathbf{1}\right)\right]^{-1}
\end{align}
and boundaries $a_{0}, a_{1}$ come from uniform prior


\subsection{$\mu$}
Prior is normal
\begin{equation}
	\sim N\left(\mu_{0}, \sigma_{0}^{2}\right)
\end{equation}
Posterior is normal
\begin{equation}
	\sim N\left(\Psi_{\mu} V_{\mu}, \Psi_{\mu}\right)
\end{equation}
where
\begin{align}
&V_{\mu}=\frac{\mu_{0}}{\sigma_{0}^{2}}+(1-\alpha) \mathbf{1}^{\mathrm{T}} \mathbf{\Sigma}^{-1}\left[\sum_{k=1}^{\kappa}\left(\mathbf{T}_{k}-\alpha \mathbf{T}_{k-1}\right)\right] \\
&\Psi_{\mu}=\left(\frac{1}{\sigma_{0}^{2}}+\kappa(1-\alpha)^{2} \mathbf{1}^{\mathrm{T}} \mathbf{\Sigma}^{-1} \mathbf{1}\right)^{-1}
\end{align}

\subsection{$\sigma^2$}
Prior is inverse-gamma
\begin{equation}
	\sim InvGamma \left(\lambda_{T}, \nu_{T}\right)
\end{equation}
Posterior is inverse-gamma
\begin{equation}
\sim InvGamma\left(\lambda_{T}+\frac{N_{A} \kappa}{2},
\nu_{T}+\frac{1}{2} \sum_{k=1}^{\kappa} \Delta \mathbf{T}_{k, k-1}^{\mathrm{T}} \mathbf{R}^{-1} \Delta \mathbf{T}_{k, k-1}\right)
\end{equation}
where
\begin{align}
\Delta \mathbf{T}_{k, k-1}&=\left(\mathbf{T}_{k}-\alpha \mathbf{T}_{k-1}-(1-\alpha) \mu \mathbf{1}\right)\\
\mathbf{R}_{i j}&=\exp \left(-\phi\left|\mathbf{x}_{i}-\mathbf{x}_{j}\right|\right)
\end{align}

\subsection{$\phi$}
Prior is log normal
\begin{equation}
	\sim logNormal\left(\mu_{\phi}, \sigma_{\phi}^{2}\right)
\end{equation}
Posterior is not analytically distributed. Its probability density function is 
\begin{equation}
P(\phi \mid \cdot) \propto P(\phi) \cdot|\mathbf{R}|^{-\kappa / 2} \cdot \exp \left(-\frac{1}{2 \sigma^{2}} \sum_{k=1}^{\kappa} \Delta \mathbf{T}_{k, k-1}^{\mathrm{T}} \mathbf{R}^{-1} \Delta \mathbf{T}_{k, k-1}\right)
\end{equation}

Since the prior of $\phi$ is log normal, it is more convenient to sample from $\log(\phi)$. We use transformation $\Phi \equiv \log(\phi)$, and posterior for $\Phi$ is
\begin{align}
P(\Phi \mid \cdot) \propto|\mathbf{R}|^{-\kappa / 2} \cdot  \exp \left(\frac{-\left(\Phi-\mu_{\phi}\right)^{2}}{2 \sigma_{\phi}^{2}}-\frac{1}{2 \sigma^{2}} \sum_{k=1}^{\kappa} \Delta \mathbf{T}_{k, k-1}^{\mathrm{T}} \mathbf{R}^{-1} \Delta \mathbf{T}_{k, k-1}\right)
\end{align}

We use Metropolis sampling in this step, and trial values are normal distributions
\begin{equation}
\Phi^{*} \mid \Phi_{t-1} \sim N\left(\Phi_{t-1}, \sigma_{\Phi, M H}^{2}\right)
\end{equation}

\subsection{$\tau_I^2$}
Prior is inverse-gamma
\begin{equation}
	\sim InvGamma \left(\lambda_{I}, \nu_{I}\right)
\end{equation}
Posterior is inverse-gamma
\begin{align}
\sim  InvGamma \left(\lambda_{I}+\frac{M_{I}}{2}, \nu_{I}+\frac{1}{2} \sum_{k=1}^{k} \mathbf{r}_{I, k}^{\mathrm{T}} \mathbf{r}_{I, k}\right)
\end{align}
where
\begin{equation}
\mathbf{r}_{I, k}=\mathbf{W}_{I, k}-\left[\mathbf{H}_{k} \mathbf{T}_{k}+\mathbf{B}_{k}\right]_{I}
\end{equation}

\subsection{$\tau_P^2$}
Prior is inverse-gamma
\begin{equation}
	\sim InvGamma \left(\lambda_{P}, \nu_{P}\right)
\end{equation}
Posterior is inverse-gamma
\begin{equation}
\sim  invGamma \left(\lambda_{P}+\frac{M_{P}}{2}, \nu_{P}+\frac{1}{2} \sum_{k=1}^{\kappa} \mathbf{r}_{P, k}^{\mathrm{T}} \mathbf{r}_{P, k}\right)
\end{equation}
where
\begin{equation}
\mathbf{r}_{P, k}=\mathbf{W}_{P, k}-\left[\mathbf{H}_{k} \mathbf{T}_{k}+\mathbf{B}_{k}\right]_{P}
\end{equation}

\subsection{$\beta_1$}
Prior is normal
\begin{equation}
	\sim N\left(\eta_{1}, \delta_{1}^{2}\right)
\end{equation}
where
\begin{equation}
	\eta_{1}=\left[\frac{\left(1-\tau_{P}^{2}\right)\left(1-\alpha^{2}\right)}{\sigma^{2}}\right]^{-1 / 2}
\end{equation}
Posterior is normal
\begin{equation}
	\sim N\left(\Psi_{\beta_{1}} V_{\beta_{0}}, \Psi_{\beta_{1}}\right)
\end{equation}
where
\begin{align}
&V_{\beta_{1}}=\frac{\eta_{1}}{\delta_{1}^{2}}+\frac{1}{\tau_{P}^{2}} \sum_{k=1}^{\kappa} \mathbf{T}_{k}^{\mathrm{T}}\left(\mathbf{W}_{P, k}-\beta_{0} \mathbf{1}_{N_{P, k}}\right) \\
&\Psi_{\beta_{1}}=\left(\frac{1}{\delta_{1}^{2}}+\frac{1}{\tau_{P}^{2}} \sum_{k=1}^{\kappa} \mathbf{T}_{P, k}^{\mathrm{T}} \mathbf{T}_{P, k}\right)^{-1}
\end{align}

\subsection{$\beta_0$}
Prior is normal
\begin{equation}
	\sim N\left(\eta_{0}, \delta_{0}^{2}\right)
\end{equation}
where $\eta_{0}=-\mu_0\eta_{1}$ is the negative of the product of prior means
for $\mu$ and $\beta_1$\\
Posterior is normal
\begin{equation}
	\sim N\left(\Psi_{\beta_{0}} V_{\beta_{0}}, \Psi_{\beta_{0}}\right)
\end{equation}
where
\begin{align}
&V_{\beta_{0}}=\frac{\eta_{0}}{\delta_{0}^{2}}+\frac{1}{\tau_{P}^{2}} \sum_{k=1}^{\kappa} \mathbf{1}_{N_{P, k}}^{\mathrm{T}}\left(\mathbf{W}_{P, k}-\beta_{1} \mathbf{H}_{P, k} \mathbf{T}_{k}\right) \\
&\Psi_{\beta_{0}}=\left(\frac{1}{\delta_{0}^{2}}+\frac{M_{P}}{\tau_{P}^{2}}\right)^{-1}
\end{align}

\end{document}
